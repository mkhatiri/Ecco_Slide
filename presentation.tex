% Copyright 2004 by Till Tantau <tantau@users.sourceforge.net>.
%
% In principle, this file can be redistributed and/or modified under
% the terms of the GNU Public License, version 2.
%
% However, this file is supposed to be a template to be modified
% for your own needs. For this reason, if you use this file as a
% template and not specifically distribute it as part of a another
% package/program, I grant the extra permission to freely copy and
% modify this file as you see fit and even to delete this copyright
% notice. 

\documentclass{beamer}

% There are many different themes available for Beamer. A comprehensive
% list with examples is given here:
% http://deic.uab.es/~iblanes/beamer_gallery/index_by_theme.html
% You can uncomment the themes below if you would like to use a different
% one:
%\usetheme{AnnArbor}
%\usetheme{Antibes}
%\usetheme{Bergen}
%\usetheme{Berkeley}
%\usetheme{Berlin}
%\usetheme{Boadilla}
%\usetheme{boxes}
%\usetheme{CambridgeUS}
%\usetheme{Copenhagen}
%\usetheme{Darmstadt}
%\usetheme{default}
%\usetheme{Frankfurt}
%\usetheme{Goettingen}
%\usetheme{Hannover}
%\usetheme{Ilmenau}
%\usetheme{JuanLesPins}
%\usetheme{Luebeck}
\usetheme{Madrid}
%\usetheme{Malmoe}
%\usetheme{Marburg}
%\usetheme{Montpellier}
%\usetheme{PaloAlto}
%\usetheme{Pittsburgh}
%\usetheme{Rochester}
%\usetheme{Singapore}
%\usetheme{Szeged}
%\usetheme{Warsaw}
\usepackage{xcolor}
\setbeamercovered{transparent}
\usepackage{amsmath}
\usepackage{eucal}

\defbeamertemplate*{footline}{mytheme}
{
    \leavevmode%
          \hbox{%
              \begin{beamercolorbox}[wd=.3\paperwidth,ht=2.25ex,dp=1ex,center]{section in head/foot}%
                  \usebeamerfont{author in head/foot}~{Mohammed~Khatiri}
              \end{beamercolorbox}%
              \begin{beamercolorbox}[wd=.4\paperwidth,ht=2.25ex,dp=1ex,center]{section in head/foot}%
                  \usebeamerfont{author in head/foot}~\insertshorttitle{}
              \end{beamercolorbox}%
              \begin{beamercolorbox}[wd=.3\paperwidth,ht=2.25ex,dp=1ex,right]{section in head/foot}%
                  \usebeamerfont{date in head/foot}
                  \insertframenumber{} / \inserttotalframenumber\hspace*{2ex} 
              \end{beamercolorbox}}%
                  \vskip0pt%
              }
\usebeamertemplate{mytheme}



\title{A new analysis of Work Stealing with latency}

% A subtitle is optional and this may be deleted
\subtitle{Work Stealing with latency}

\author{Nicolas~Gast\inst{1} \and \textcolor{red}{Mohammed~Khatiri}\inst{1}\inst{2} \and Denis~Trystram\inst{1} \and Frédéric~Wagner\inst{1}}
% - Give the names in the same order as the appear in the paper.
% - Use the \inst{?} command only if the authors have different
%   affiliation.

\institute[Univ. Grenoble Alpes, CNRS, Inria, LIG] % (optional, but mostly needed)
{
    \inst{1}%
  Univ. Grenoble Alpes\\
  CNRS, Inria, LIG, France
  \and
  \inst{2}%
  University Mohammed First\\
  Faculty of Sciences, LaRI, Morocco}
% - Use the \inst command only if there are several affiliations.
% - Keep it simple, no one is interested in your street address.

\date{ECCO 2018}
% - Either use conference name or its abbreviation.
% - Not really informative to the audience, more for people (including
%   yourself) who are reading the slides online

%\subject{Theoretical Computer Science}
% This is only inserted into the PDF information catalog. Can be left
% out. 

% If you have a file called "university-logo-filename.xxx", where xxx
% is a graphic format that can be processed by latex or pdflatex,
% resp., then you can add a logo as follows:

% \pgfdeclareimage[height=0.5cm]{university-logo}{university-logo-filename}
% \logo{\pgfuseimage{university-logo}}

% Delete this, if you do not want the table of contents to pop up at
% the beginning of each subsection:
\AtBeginSubsection[]
{
    \begin{frame}<beamer>{Outline}
        \tableofcontents[currentsection,currentsubsection]
    \end{frame}
}

% Let's get started
\begin{document}

\begin{frame}
    \titlepage
\end{frame}

\begin{frame}{Outline}
    \tableofcontents
    % You might wish to add the option [pausesections]
\end{frame}

% Section and subsections will appear in the presentation overview
% and table of contents.
\section{Context}

\subsection{Objectives}

\begin{frame}{Objectives of the talk}
    \begin{itemize}[<+->]
        \item {
               Study the Work Stealing algorithm in a Distributed-memory plates-form where communications matter. 
            }\\ 
        \item {

               Create new realistic model for distributed-memory cluster of $p$ identical processors including communication latency. 
            }\\          

        \item {
                Provide an upper bound of the expected makespan
            }
    \end{itemize}
\end{frame}


\begin{frame}{Scheduling}
    \begin{block}{Problem}
        The scheduling problem is to answer the two following questions
        for each of the n independent tasks which represents an application.
        \begin{itemize}
            \item \alert{Where?} – determine the computing resources for executing each task.
            \item \alert{When?} – determined when each task will begin. 

        \end{itemize}
    \end{block}
    How schedulers $n$ tasks on $p$ processors ?
    \begin{itemize}
        \item \textbf{WHERE is crucial} : The load between resources must be balanced. 
        \item Need a dynamic algorithm to determine such an assignment.

    \end{itemize}
\end{frame}



%\subsection{The classical problem}

% You can reveal the parts of a slide one at a time
% with the \pause command:
\begin{frame}{The classical problem}
    \begin{itemize}
        \item { Workload $\mathcal{W}$ of $n$ unit independent tasks.}
        \item { \emph{p} identical processors in distributed-memory cluster.  }                                     
        \item { Each processor has its private memory}
        \item { Processors are linked by an interconnection network}                                     
    \end{itemize}
        
        \begin{block}{Makespan}
            The optimized objective is the completion time of the $n$ tasks Which called the \alert{Makespan} and denoed by \alert{$C_{max}$}.
        \end{block}
        
     \begin{itemize}
         \item \alert{ Communication cost between processors is most often ignored. }                                     
    \end{itemize}

\end{frame}

\begin{frame}{Work Stealing algorithm}

 \begin{itemize}
     \item \alert{Decentralized list scheduling algorithm.}
         \pause
        \item { Each processor has its own queue of tasks}                                     
            \pause
        \item { An active processor (non-empty tasks queue) executes tasks from its queue.}
            \pause
        \item { An idle processor with empty queue (Stealer) randomly chooses another processor (victim) to steal some work:}                                     
            \pause
            \begin{itemize}
                \item {If the victim’s queue is non empty, the thief steals half of the tasks and resumes execution at the next time slot}
                    \pause
                \item {Otherwise, the thief tries again at the next time slot}
            \end{itemize}
        \pause 
        \item {If several thieves target the same victim a random succeed, others fail.} 

    \end{itemize}
\end{frame}


\subsection{Previous Work}

\begin{frame}{Previous Work}
    \color{blue} {[Blumofe and Leiserson 1999, Arora et al. 2001  ]}
        \begin{itemize}        
            \item Bound the expected Makespan of a series-parallel precedence graph with $\mathcal{W}$ unit tasks on $p$ processors
             \item $E(C_{\max}) \leq \frac{\mathcal{W}}{p}+\mathcal{O}(D)$  Where $D$ is the length of the critical path of the graph.
        \end{itemize}  
    \color{blue} [Tchiboukdjian et al. 2013] 
         \begin{itemize}        
             \item $\mathcal{W}$ unit independent tasks on $p$ processors (communication cost is ignored).  
             \item $E(C_{\max}) \leq \frac{\mathcal{W}}{p}+c.(\log_2 \mathcal{W})+\Theta(1)$ where $c=3.24$ 
        \end{itemize}  

        \alert{Need a precise analysis of work stealing when communication is no-ignored}
\end{frame}

\section{Extend to Work Stealing with communications cost}

\begin{frame}{problem}
    \begin{alertblock}{ }
        
    \end{alertblock}
  \begin{itemize}
        \item {workload $\mathcal{w}$ of $n$ unit independent tasks.}
        \item {\emph{p} identical processors in distributed-memory cluster
            \alert{with explicit communication}.}
        \item {processors are linked by an interconnection network}
        \item {to simplify the processors are linked by an interconnection network}


    \end{itemize}
\end{frame}

\begin{frame}{Model with communications cost (latency)}

\end{frame}

\begin{frame}{Analysis}
\end{frame}

\begin{frame}{Experimental Results}

\end{frame}




\begin{frame}{Conclusion and Perspective}
    Conclusion
  \begin{itemize}
        \item 
        \item 
        \item 
    \end{itemize}
       
   Perspective
  \begin{itemize}
        \item 
        \item 
        \item 
    \end{itemize}
\end{frame}






% Placing a * after \section means it will not show in the
% outline or table of contents.
\section*{Summary}


\end{document}



